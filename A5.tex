\documentclass[11pt, answers]{exam}
\renewcommand{\baselinestretch}{1.05}
\usepackage{amsmath,amsthm,verbatim,amssymb,amsfonts,amscd, graphicx}
\usepackage{graphics}

\usepackage{afterpage}
\usepackage{caption}

\usepackage{tikz}
\usepackage{fancybox}

\usepackage{clrscode3e}

\topmargin0.0cm
\headheight0.0cm
\headsep0.0cm
\oddsidemargin0.0cm
\textheight23.0cm
\textwidth16.5cm
\footskip1.0cm
\theoremstyle{plain}
\newtheorem{theorem}{Theorem}
\newtheorem{corollary}{Corollary}
\newtheorem{lemma}{Lemma}
\newtheorem{proposition}{Proposition}
\newtheorem*{surfacecor}{Corollary 1}
\newtheorem{conjecture}{Conjecture}  
\theoremstyle{definition}
\newtheorem{definition}{Definition}

 \begin{document}
 


\title{CSC263: Assignment 3}
\date{March 30th, 2017}
\author{Junjie Cheng, Jiayun Liu, Zi Hao Lin}
\maketitle

\unframedsolutions

\begin{questions}
\question
%Question1
\begin{solution}
\begin{parts}
\part

\end{parts}
\end{solution}

\question
%Question2
\begin{solution}
\begin{parts}
\part 
Assume there exist a bipartite G that contains a simple circle of odd length.

Start with any vertex and name each vertex in this circle in order with $v_1$,$v_2$...$v_{2k+1}$. Every two adjacent vertexes are connected by an edge, and since it is a circle, $v_{2k+1}$ is connected with $v_1$.

For all $v_i$, it will be put into V1 if i is odd and V2 if i is even. As a result, until $v_{2k}$, no two connected vertexes will be put into the same subset.
 
However, $v_{2k+1}$ will inevitably be put into V1, which is the same set of $v_1$. By the principle of bipartite, no two connected vertexes can be put into the same V, so there is a contradiction.

Thus, a bipartite G contains no simple circle of odd length.

\part
General idea: We are going to perform a BFS and use the distance. If d[v] is odd, then put v into V1, if d[v] is even, then put v into V2.

Proof:

 \textbf{Base case:}

If v is the only vertex with d[v], then d[v] = d[s] = 0, put it into V2. 

If v and u are the only two vertexes and v discovers u, then d[v] = 0, d[u] = d[v] + 1 = 1. As a result, v is in V2, u is in V1, and the two connected vertexes are in different subsets. This satisfies the principle of bipartite.

\textbf{Induction:}

\textbf{I.H:} 

Suppose all vertexes before v (include v) are put into V1 and V2 and satisfy the condition of bipartite. Assume every circle in G has even length and v connect to u.

\textbf{I.S:}

By the principle of BFS, as u is connect to v, then:

\textbf{Case 1}: u is white

Then d[u] = d[v] + 1. 

Then if d[v] is odd, then d[u] is even, vice versa.

Then u and v will be in two different V, which satisfy the bipartite condition for any v and u.

\textbf{Case 2}: u is gray or black

Now we meet a circle. 

As every circle given has even length, if we start from u to v, by principle of BFS, d[u] = d[v] + 2k +1 where k is the number of vertexes in the circle.

So u and v will still be in different V, the graph with u, v is bipartite.

Induction follows.

\part
The algorithm:

Run BFS and put vertexes into V1 and V2 according to part(b).

Every time when v discovers u and u is grey or black, check if (d[v] - d[u] - 1) $\%$ 2 = 0.

If true, then we meet an even circle, keep going.

If false, then we meet an odd circle, return "Not bipartite".


Running time:

In the algorithm, both putting vertexes into subset or checking even or odd circles only need some simple steps and run constant time.

Therefore, the running time will be same as BFS(G,s), which is O(m+n).




\end{parts}
\end{solution}


\question
%Question3
\begin{solution}
\begin{parts}
\part


\end{parts}
\end{solution}

\end{questions}
\end{document}
